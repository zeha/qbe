%%
%% Qbe SAS SystemDocumentation
%% (C) Copyright 2001-2004 Christian Hofstaedtler
%%
%% $Id: part-03.tex 2 2004-03-10 08:40:26Z ch $
%%

\cChapter{Webserver Integration}
Qbe SAS ist vorbereitet um in Kooperation mit einem getrennten Webserver die Webseite der Institution und die pers�nlichen Seiten der Systembenutzer anzuzeigen.

\section{Software}
Folgende Standardsoftware wird verwendet:

\begin{Verbatim}
Debian GNU/Linux woody, RedHat Linux, alternativ FreeBSD/OpenBSD
Apache HTTP Server 1.3 oder 2.0
NFS Client
\end{Verbatim}

\section{Konfiguration}
Da keine spezielle Software verwendet wird, beschr�nkt sich die Konfiguration auf das NFS Filesystem und den Apache Webserver.

Notiz: Es ist nicht notwendig eine LDAP-Benutzerauthentifizierung einzurichten. Idealerweise gibt es auf dem Webserver nur Systemaccounts und einen Systemverwalter (nicht \verb|root|). \verb|root| sollte sich nicht remote anmelden k�nnen.

\subsection{NFS}
Datei \verb|/etc/fstab| muss um folgenden Eintrag (in einer Zeile) erg�nzt werden:

\begin{Verbatim}
10.0.2.10:/export/homes /import/homes           nfs     rw,soft,
timeo=60,async,nodev,noexec,nouser,nosuid 0 0
\end{Verbatim}

Dies wei�t das System an, beim Neustart automatisch das Filesystem mit den Benutzerverzeichnissen via NFS vom AuthServer (hier: 10.0.2.10) zu importieren. Zus�tzlich werden einige Parameter gesetzt die die Geschwindigkeit und Sicherheit positiv beeinflussen.

\subsection{Apache httpd}
In der Apache Konfigurationsdatei \verb|httpd.conf| muss folgendes sinngem�� hinzugef�gt werden (Beispiel f�r Apache 1.3):
\begin{Verbatim}
LoadModule userdir_module     modules/mod_userdir.so

<IfModule mod_userdir.c>
    UserDir disabled root
    UserDir /import/homes/*/web
</IfModule>
<Directory /import/homes/*/web>
    AllowOverride All
    Options Indexes Includes
    Order allow,deny
    Allow from all
</Directory>
\end{Verbatim}

Soll auch die Institutsseite am AuthServer abgelegt werden, kann diese im Benutzerverzeichniss des Benutzers "`web"' geschehen. Dazu muss zus�tzlich im \verb|httpd.conf| eingetragen werden:
\begin{Verbatim}
DocumentRoot "/import/homes/web"
\end{Verbatim}


%% *eof*
