%%
%% Qbe SAS SystemDocumentation
%% (C) Copyright 2001-2004 Christian Hofstaedtler
%%
%% $Id: part-03.tex 36 2004-05-12 13:04:34Z ch $
%%

\cChapter{Webserver Integration}
Qbe SAS ist vorbereitet um in Kooperation mit einem getrennten Webserver, die Webseite der Institution und die pers�nlichen Seiten der Systembenutzer anzuzeigen.

\section{Software}
Folgende Standardsoftware wird verwendet:

\begin{itemize}
\item Debian GNU/Linux woody oder OpenBSD 3.2+ oder FreeBSD 4.x
\item Apache \index{HTTP}{HTTP} Server 1.3 oder 2.0
\item NFS Client
\end{itemize}

\section{Konfiguration}
Da keine spezielle Software verwendet wird, beschr�nkt sich die Konfiguration auf das NFS Filesystem und den Apache Webserver.

Notiz: Es ist nicht notwendig, eine LDAP-Benutzerauthentifizierung einzurichten. Idealerweise gibt es auf dem Webserver nur Systemaccounts und einen Systemverwalter (nicht \verb|root|). \verb|root| sollte sich nicht �ber das Netzwerk anmelden k�nnen.

\subsection{NFS}
Datei \verb|/etc/fstab| muss um folgenden Eintrag (in einer Zeile) erg�nzt werden:

\begin{lstlisting}
10.0.2.100:/export/homes /import/homes           nfs     rw,soft,timeo=60,async,nodev,noexec,nouser,nosuid 0 0
\end{lstlisting}

Dies weist das System an, beim Neustart automatisch das Dateisystem mit den Benutzerverzeichnissen via NFS vom AuthServer (hier: 10.0.2.100) zu importieren. Zus�tzlich werden einige Parameter gesetzt die die Geschwindigkeit und Sicherheit positiv beeinflussen.

\subsection{Apache httpd}
In der Apache Konfigurationsdatei \verb|httpd.conf| muss folgendes sinngem�� hinzugef�gt werden (Beispiel f�r Apache Version 1.3):

\begin{lstlisting}[language=]
; Modul fuer Benutzerverzeichnisse laden
LoadModule userdir_module     modules/mod_userdir.so

; Fuer root kein Benutzerverzeichnis, fuer alle anderen in ~/web/
<IfModule mod_userdir.c>
    UserDir disabled root
    UserDir /import/homes/*/web
</IfModule>
<Directory /import/homes/*/web>
    AllowOverride All
    Options Indexes Includes
    Order allow,deny
    Allow from all
</Directory>
\end{lstlisting}

Soll auch die Institutionsseite am AuthServer abgelegt werden, kann diese im Benutzerverzeichniss des Benutzers "web" geschehen. 
Dazu muss im \verb|httpd.conf| zus�tzlich folgendes eingetragen werden:
\begin{lstlisting}
DocumentRoot "/import/homes/web"
\end{lstlisting}


%% *eof*

