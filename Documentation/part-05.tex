%%
%% Qbe SAS SystemDocumentation
%% (C) Copyright 2001-2004 Christian Hofstaedtler
%%
%% $Id: part-04.tex 24 2004-05-07 06:28:49Z ch $
%%

\cChapter{Client/Server Protokoll}
\label{chap-csprotocol}

Der Qbe SAS Client und der Qbe Authentication Server benutzen eine eigene, einfache \index{RPC}{RPC} Implementation, die auf HTTP aufsetzt. Verbindungen vom Server zum Client sind immer unverschl�sselt, in die andere Richtung steht es dem Client frei zwischen \index{HTTP}{HTTP} und \index{HTTP-SSL}{HTTP-SSL} zu w�hlen.

\section{UserAgent Feldbeschreibung}
Der Qbe SAS Client sollte immer einen UserAgent-\index{Header}{Header} mitschicken, der den Client als richtigen Client mit definierter Versionsnummer identifiziert. Der Feldaufbau f�r alte Versionen ist wie folgend:
\begin{itemize}
\item Zeichenkette "iLogin"
\item Leerzeichen
\item Der Programmname. Zum Beispiel "QbeService (WIN32)"
\item Leerzeichen
\item Qbe SAS Client Versionsnummer. Zum Beispiel "2.23.00"
\end{itemize}

Beginnend mit Client Version 2.23.10 wird folgendes Format verwendet:
\index{QbeService}

\begin{itemize}
\item Zeichenkette "QbeService/"
\item Qbe SAS Client Versionsnummer. Zum Beispiel "2.23.89"
\end{itemize}

Clients, die nicht diesen Anforderungen entsprechen, werden eventuell vom Server abgewiesen. 
Die neuen Clients melden den Betriebssystemnamen nicht mehr an den Server.

\section{Server API}
Folgende Befehle sind am Server zur Benutzung durch den Client implementiert:

\subsection{/rpc/client/login}
�berpr�ft den Benutzernamen und das Passwort gegen�ber dem \index{LDAP}{LDAP} Verzeichnis und tr�gt dann die aktuelle IP- und MAC-Adresse des Clients in das Benutzerobjekt ein. Ist bereits eine IP- oder MAC-Adresse eingetragen, werden diese �berschrieben und alte Anmeldungen werden damit ung�ltig.

Akzeptiert die GET-Parameter "user" und "pass" f�r Testzwecke, oder vom Client die \index{Header}{Header} "iLogin-User" (Benutzername), "iLogin-Pass" (Passwort), oder "iLogin-Token" (Benutzername/Passwort base64 enkodiert). Auf jeden Fall muss der GET-Paramter "ver" (die Client-Versionsnummer) und der "UserAgent"-Header vorhanden sein.

Die Antwort besteht aus dem \index{HTTP}{HTTP} Statuscode, sowie dem Internet-Status aus dem LDAP ("iLogin-User-State"), dem bisherigen Internettraffic ("iLogin-Stats-Traffic") und der Speicherplatznutzung ("iLogin-Stats-Disk"). 

Der HTTP \index{Statuscode}{Statuscode} kann folgende Werte annehmen:
\begin{itemize}
\item 200 OK, Anmeldung erfolgreich
\item 401 Fehler: Benutzer muss zuerst Passwort �ndern
\item 403 Fehler: Benutzername/Passwort falsch
\item 404 Fehler: Das API hat sich ge�ndert, bzw. der Qbe SAS Client ist veraltet.
\item 500 Fehler: Allgemeiner Serverfehler. Client soll die Anmeldung nicht erneut versuchen.
\end{itemize}

Die Anmeldeversuche und Resultate werden mittels \index{syslog}{syslog} nach /var/log/qbe.log gespeichert.

\subsection{/rpc/client/logout}
Meldet den aktuellen Benutzer vom System ab. Akzeptiert keine Parameter und liefert immer HTTP-Statuscode 200 ("OK") zur�ck.

Zur Abmeldung wird nach dem ersten LDAP-Eintrag, der auf den folgender Filter passt gesucht: \textit{(loggenonHost=\$ClientIP)}. In diesem Eintrag werden dann die Attribute loggedon\-Host, loggedon\-Mac, und last\-Activity gel�scht.

Die Abmeldungen werden mittels \index{syslog}{syslog} nach /var/log/qbe.log gespeichert.

\subsection{/rpc/client/svc-top.php}
Zeigt den Schriftzug "Qbe SAS Client MJ.MN.RV" (wobei MJ = Majorversion, MN = Minorversion, RV = Revision des Clients) auf blauem Hintergrund. 

Ab Qbe Application Server Version 0.91 wird zus�tzlich auch ein Hinweis auf den Serverstatus angezeigt. z.B: Server: ok. Andere Zust�nde k�nnen fail oder critical sein.

\subsection{/rpc/client/svc-frame.php}
Erstellt ein HTML Frameset zur Verwendung in der Windows-Version des Qbe SAS Clients. Enth�lt ein Top-Frame mit dem \verb|svc-top.php| und ein Content-Frame mit der QbeSvc-URL "http://127.0.0.1:7666/web/menu".
Wird vom Client im Parameter \verb|ver| eine Versionsnummer �bermittelt und entspricht diese nicht der aktuellen Version, wird statt dem Content Frame die Update-Info Seite \verb|update.php| angezeigt.

\subsection{/rpc/client/update.php}
Weist auf eine neue verf�gbare Qbe SAS Client Version hin. Auf PCs mit installiertem \index{HDGuard}{HDGuard} wird der Hinweis mittels einem VBScript/Registry-Check �bersprungen.

\section{Client API}

Die Client-Befehle sind unter Windows im QbeSvc.EXE, im Xplat Client im QbeService.EXE implementiert. Befehle werden nur von IPs im Bereich 10.0.2.0/24 akzeptiert, der Server muss sich jedoch nicht weiter ausweisen.

Befehle mit Pr�fix "/web" sollten nicht vom Server, sondern nur intern vom Qbe SAS Client benutzt werden.

Der Client antwortet im Fehlerfall mit \index{Statuscode}{Statuscode} 400 oder 404. Code 400 wird f�r nicht implementierte Methoden (HEAD, POST) verwendet, 404 f�r nicht implementierte Pfade.

\subsection{/}
Optional: Weiterleitung nach "/web/menu".

\subsection{/web/menu}
Optional: Zeigt eine einfache �bersicht �ber den Status des Qbe SAS Client zur lokalen Verwendung an. �blicherweise werden der Benutzername, der Servername, Zeit der letzten Verbindung, eventuell der letzte Fehler bzw. die Benutzerstatistiken angezeigt.

\subsection{/web/login}
Optional: Zeigt das Anmeldeformular zur lokalen Verwendung an. Mittels ActiveX Objekt wird ein eventuell vorher gespeichertes Benutzername/Passwort-Paar aus der Registry ausgelesen und abgeschickt. Andernfalls kann man das Paar abspeichern.

Das Formular f�hrt einen GET-Request auf "/auth/setlogin" aus.

\subsection{/web/hta-login}
Optional: Hilfsdaten f�r den Windows Client.

\subsection{/web/hta-login-done}
Optional: Hilfsdaten f�r den Windows Client.

\subsection{/web/hta-login-post}
Optional: Hilfsdaten f�r den Windows Client.

\subsection{/web/cleardata}
Optional: L�scht ein eventuell vorher gespeichertes Benutzername/Passwort-Paar via ActiveX-Objekt aus der Registry.

\subsection{/web/logout}
Optional: Initiiert eine Abmeldung im Hintergrund.

\subsection{/auth/login}
Initiiert eine Hintergrundanmeldung.

\subsection{/auth/setauthserver}
Optional: Akzeptiert im GET-Parameter "server" einen neuen Namen f�r den Qbe Authentication Server. Default: "\index{qbe-auth}{qbe-auth}"

\subsection{/auth/setlogin}
Akzeptiert die GET-Daten des Formulars "/web/login" und initiiert eine Hintergrundanmeldung.

\subsection{/auth/logout}
Initiiert eine Abmeldung im Hintergrund.

\subsection{/auth/forcerefresh}

\subsection{/service/stop}
Optional: Windows-Spezifisch: Beendet den QbeSvc.

\subsection{/system/exec}
Optional: Windows-Spezifisch: F�hrt einen Befehl im Kontext des QbeSvc aus.

\subsection{/system/setmanager}
�bernimmt den GET-QueryString als neue zus�tzliche Manager-IP, von der Befehle entgegen genommen werden.

\subsection{/system/getinfo}
Akzeptiert den GET-Parameter "type" und liefert die gew�nschte Information zur�ck. G�ltige Werte sind:

\begin{description}
\item[osversion] Optional: Die Betriebssystemversion.
\item[osregowner] Optional: Der Name des Benutzers auf den das Betriebssystem eingetragen wurde.
\item[hostname] Der Computername.
\item[username] Der Qbe SAS Benutzername oder "*VOID*" wenn kein Benutzername bekannt ist.
\item[authtoken] Das Qbe SAS Benutzername/Passwort-Token oder "*VOID*" wenn kein Benutzername/Passwort-Token bekannt ist.
\item[version] Die Qbe SAS Clientversion.
\item[cvsid] Optional: Die CVS bzw. SVN \$Id\$.
\item[copyright] Optional: Die Copyright-Information des installierten Qbe SAS Client.
\item[time] Optional: Die aktuelle Zeit.
\item[authserver] Der verwendete Qbe Authentication Server
\item[connectionstate] Optional: Der zuletzt bekannte Zustand der Systemverbindung.
\item[internetstate] Optional: Der zuletzt bekannte Internet-Zustand des Benutzers.
\end{description}

\subsection{/system/message}
Sendet den GET-QueryString als Nachricht an den Benutzer.

\subsection{/system/shutdown}
Optional: F�hrt den Computer herunter.

\subsection{/system/restart}
Optional: Startet den Computer neu.

\subsection{/ilogin/update}
Server best�tigt eine Anmeldung oder sendet einen Keep-Alive-Request. Gleichzeitig werden der Internetstatus, die Internetnutzung in Prozent und die Speicherplatznutzung in Prozent �bertragen.

\subsection{/ilogin/logout}
Server best�tigt eine Abmeldung. Client l�scht alle relevanten Daten wie Benutzername, Passwort, letzte Kommunikationszeit, Internetstatus \ldots

\subsection{/ilogin/time}
Optional: Server sendet die aktuelle Zeit in Sekunden seit 1. 1. 1970 an den Client. Dieser sollte dann die Systemzeit richtig einstellen.

\placefigx{flow-client-login}{Ablaufdiagramm Client Anmeldung}{width=14.5cm}

%% *eof*
