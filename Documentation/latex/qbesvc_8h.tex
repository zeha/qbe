\hypertarget{qbesvc_8h}{
\section{qbesvc.h Dateireferenz}
\label{qbesvc_8h}\index{qbesvc.h@{qbesvc.h}}
}


\subsection*{Makrodefinitionen}
\begin{CompactItemize}
\item 
\#define \hyperlink{qbesvc_8h_a0}{UNICODE}
\item 
\#define \hyperlink{qbesvc_8h_a1}{\_\-UNICODE}
\end{CompactItemize}
\subsection*{Funktionen}
\begin{CompactItemize}
\item 
\hyperlink{QbeGina_8h_a2}{DWORD} \hyperlink{qbesvc_8h_a2}{qbe\_\-qbesvc\_\-login} (LPWSTR sz\-Username, LPWSTR sz\-Password)
\begin{CompactList}\small\item\em UEbergibt sz\-Username und sz\-Password via HTTP an den lokalen Qbe\-Svc. \item\end{CompactList}\item 
\hyperlink{QbeGina_8h_a2}{DWORD} \hyperlink{qbesvc_8h_a3}{qbe\_\-qbesvc\_\-logout} ()
\begin{CompactList}\small\item\em Initiiert im Qbe\-Svc die Abmeldung des aktiven Benutzers. \item\end{CompactList}\end{CompactItemize}


\subsection{Makro-Dokumentation}
\hypertarget{qbesvc_8h_a1}{
\index{qbesvc.h@{qbesvc.h}!_UNICODE@{\_\-UNICODE}}
\index{_UNICODE@{\_\-UNICODE}!qbesvc.h@{qbesvc.h}}
\subsubsection[\_\-UNICODE]{\setlength{\rightskip}{0pt plus 5cm}\#define \_\-UNICODE}}
\label{qbesvc_8h_a1}




Definiert in Zeile 3 der Datei qbesvc.h.\hypertarget{qbesvc_8h_a0}{
\index{qbesvc.h@{qbesvc.h}!UNICODE@{UNICODE}}
\index{UNICODE@{UNICODE}!qbesvc.h@{qbesvc.h}}
\subsubsection[UNICODE]{\setlength{\rightskip}{0pt plus 5cm}\#define UNICODE}}
\label{qbesvc_8h_a0}




Definiert in Zeile 2 der Datei qbesvc.h.

\subsection{Dokumentation der Funktionen}
\hypertarget{qbesvc_8h_a2}{
\index{qbesvc.h@{qbesvc.h}!qbe_qbesvc_login@{qbe\_\-qbesvc\_\-login}}
\index{qbe_qbesvc_login@{qbe\_\-qbesvc\_\-login}!qbesvc.h@{qbesvc.h}}
\subsubsection[qbe\_\-qbesvc\_\-login]{\setlength{\rightskip}{0pt plus 5cm}\hyperlink{QbeGina_8h_a2}{DWORD} qbe\_\-qbesvc\_\-login (LPWSTR {\em sz\-Username}, LPWSTR {\em sz\-Password})}}
\label{qbesvc_8h_a2}


UEbergibt sz\-Username und sz\-Password via HTTP an den lokalen Qbe\-Svc. 



Definiert in Zeile 11 der Datei qbesvc.c.

Benutzt DWORD und Write\-Log\-File().

Wird benutzt von Wlx\-Logged\-Out\-SAS().



\footnotesize\begin{verbatim}12 {
13   LPWSTR lpUrl;
14   DWORD dwSecSize;
15   DWORD dwSecData;
16   DWORD dFlags;
17   HINTERNET myConn;
18   HINTERNET myReq;
19   HINTERNET myInet;
20         DWORD rc = NO_ERROR;
21 
22   myInet = WinHttpOpen( TEXT("User-Agent: QbeNP/2.23"),
23                           WINHTTP_ACCESS_TYPE_NO_PROXY,
24                           WINHTTP_NO_PROXY_NAME, 
25                           WINHTTP_NO_PROXY_BYPASS, 0 );
26   
27   lpUrl = LocalAlloc(LPTR,1024);
28   
29   dFlags = WINHTTP_FLAG_REFRESH;
30   dwSecSize = sizeof(DWORD);
31   dwSecData = SECURITY_FLAG_IGNORE_CERT_CN_INVALID | SECURITY_FLAG_IGNORE_CERT_DATE_INVALID | SECURITY_FLAG_IGNORE_UNKNOWN_CA;
32 
33   myConn = WinHttpConnect( myInet, TEXT("localhost"), 7666, 0 );
34   if (myConn == NULL)
35   {
36     rc = WN_NO_NETWORK;
37     goto done;
38   }
39 
40   wsprintf(lpUrl,TEXT("/auth/setlogin?user=%s&pass=%s&source=gina"),szUsername,szPassword);
41 WriteLogFile(lpUrl);
42 WriteLogFile(TEXT("\r\n"));
43   myReq = WinHttpOpenRequest( myConn, L"GET", lpUrl, NULL, WINHTTP_NO_REFERER, WINHTTP_DEFAULT_ACCEPT_TYPES, dFlags );
44   WinHttpSetOption( myReq, WINHTTP_OPTION_SECURITY_FLAGS, &dwSecData, dwSecSize );
45   
46   if (WinHttpSendRequest( myReq, WINHTTP_NO_ADDITIONAL_HEADERS, 0, WINHTTP_NO_REQUEST_DATA, 0, 0, 0 ) == FALSE)
47   {
48     rc = WN_NO_NETWORK;
49     goto done;
50   } 
51   
52   WinHttpReceiveResponse( myReq, NULL);
53 
54   if (myReq)  WinHttpCloseHandle(myReq);
55 
56   myReq = WinHttpOpenRequest( myConn, TEXT("GET"), TEXT("/auth/login"), NULL, WINHTTP_NO_REFERER, WINHTTP_DEFAULT_ACCEPT_TYPES, dFlags );
57   WinHttpSetOption( myReq, WINHTTP_OPTION_SECURITY_FLAGS, &dwSecData, dwSecSize );
58   
59   WinHttpSendRequest( myReq, WINHTTP_NO_ADDITIONAL_HEADERS, 0, WINHTTP_NO_REQUEST_DATA, 0, 0, 0 );
60   WinHttpReceiveResponse( myReq, NULL);
61 
62 done:
63   if (myReq)  WinHttpCloseHandle(myReq);
64   if (myConn) WinHttpCloseHandle(myConn);
65   if (myInet) WinHttpCloseHandle(myInet);
66 
67   LocalFree(lpUrl);
68 
69   return rc;
70 }
\end{verbatim}\normalsize 
\hypertarget{qbesvc_8h_a3}{
\index{qbesvc.h@{qbesvc.h}!qbe_qbesvc_logout@{qbe\_\-qbesvc\_\-logout}}
\index{qbe_qbesvc_logout@{qbe\_\-qbesvc\_\-logout}!qbesvc.h@{qbesvc.h}}
\subsubsection[qbe\_\-qbesvc\_\-logout]{\setlength{\rightskip}{0pt plus 5cm}\hyperlink{QbeGina_8h_a2}{DWORD} qbe\_\-qbesvc\_\-logout ()}}
\label{qbesvc_8h_a3}


Initiiert im Qbe\-Svc die Abmeldung des aktiven Benutzers. 



Definiert in Zeile 73 der Datei qbesvc.c.

Benutzt DWORD.

Wird benutzt von Wlx\-Logoff().



\footnotesize\begin{verbatim}74 {
75   DWORD dwSecSize;
76   DWORD dwSecData;
77   DWORD dFlags;
78   HINTERNET myConn;
79   HINTERNET myReq;
80   HINTERNET myInet;
81         DWORD rc = NO_ERROR;
82 
83   myInet = WinHttpOpen( TEXT("User-Agent: QbeNP/2.23"),  
84                           WINHTTP_ACCESS_TYPE_NO_PROXY,
85                           WINHTTP_NO_PROXY_NAME, 
86                           WINHTTP_NO_PROXY_BYPASS, 0 );
87   
88   dFlags = WINHTTP_FLAG_REFRESH;
89   dwSecSize = sizeof(DWORD);
90   dwSecData = SECURITY_FLAG_IGNORE_CERT_CN_INVALID | SECURITY_FLAG_IGNORE_CERT_DATE_INVALID | SECURITY_FLAG_IGNORE_UNKNOWN_CA;
91 
92   myConn = WinHttpConnect( myInet, TEXT("localhost"), 7666, 0 );
93   if (myConn == NULL)
94   {
95     rc = WN_NO_NETWORK;
96     goto done;
97   }
98 
99   myReq = WinHttpOpenRequest( myConn, TEXT("GET"), TEXT("/auth/logout"), NULL, WINHTTP_NO_REFERER, WINHTTP_DEFAULT_ACCEPT_TYPES, dFlags );
100   WinHttpSetOption( myReq, WINHTTP_OPTION_SECURITY_FLAGS, &dwSecData, dwSecSize );
101   
102   WinHttpSendRequest( myReq, WINHTTP_NO_ADDITIONAL_HEADERS, 0, WINHTTP_NO_REQUEST_DATA, 0, 0, 0 );
103   WinHttpReceiveResponse( myReq, NULL);
104 
105 done:
106   if (myReq)  WinHttpCloseHandle(myReq);
107   if (myConn) WinHttpCloseHandle(myConn);
108   if (myInet) WinHttpCloseHandle(myInet);
109 
110   return rc;
111 }
\end{verbatim}\normalsize 
