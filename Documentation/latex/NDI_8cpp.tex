\hypertarget{NDI_8cpp}{
\section{NDI.cpp Dateireferenz}
\label{NDI_8cpp}\index{NDI.cpp@{NDI.cpp}}
}


{\tt \#include \char`\"{}stdafx.h\char`\"{}}\par
{\tt \#include $<$setupapi.h$>$}\par
{\tt \#include $<$devguid.h$>$}\par
{\tt \#include $<$objbase.h$>$}\par
{\tt \#include $<$ole2.h$>$}\par
{\tt \#include \char`\"{}C:$\backslash$WINDDK$\backslash$2600.1106$\backslash$inc$\backslash$w2k$\backslash$netcfgx.h\char`\"{}}\par
{\tt \#include \char`\"{}C:$\backslash$WINDDK$\backslash$2600.1106$\backslash$inc$\backslash$w2k$\backslash$netcfgn.h\char`\"{}}\par
{\tt \#include \char`\"{}NDI.h\char`\"{}}\par
\subsection*{Namensbereiche}
\begin{CompactItemize}
\item 
namespace \hyperlink{namespaceQbeNDI}{Qbe\-NDI}
\end{CompactItemize}
\subsection*{Funktionen}
\begin{CompactItemize}
\item 
ULONG \hyperlink{NDI_8cpp_a1}{Release\-Obj} (IUnknown $\ast$punk)
\item 
HRESULT \hyperlink{NDI_8cpp_a2}{Hr\-Get\-INet\-Cfg} (IN BOOL f\-Get\-Write\-Lock, INet\-Cfg $\ast$$\ast$ppnc)
\item 
HRESULT \hyperlink{NDI_8cpp_a3}{Hr\-Release\-INet\-Cfg} (BOOL f\-Has\-Write\-Lock, INet\-Cfg $\ast$pnc)
\item 
HRESULT \hyperlink{NDI_8cpp_a4}{Hr\-Install\-Net\-Component} (IN INet\-Cfg $\ast$pnc, IN PCWSTR sz\-Component\-Id, IN const GUID $\ast$pguid\-Class)
\item 
HRESULT \hyperlink{NDI_8cpp_a5}{Hr\-Uninstall\-Net\-Component} (IN INet\-Cfg $\ast$pnc, IN PCWSTR sz\-Component\-Id)
\end{CompactItemize}
\subsection*{Variablen}
\begin{CompactItemize}
\item 
int \hyperlink{NDI_8cpp_a0}{work\-State}
\end{CompactItemize}


\subsection{Dokumentation der Funktionen}
\hypertarget{NDI_8cpp_a2}{
\index{NDI.cpp@{NDI.cpp}!HrGetINetCfg@{HrGetINetCfg}}
\index{HrGetINetCfg@{HrGetINetCfg}!NDI.cpp@{NDI.cpp}}
\subsubsection[HrGetINetCfg]{\setlength{\rightskip}{0pt plus 5cm}HRESULT Hr\-Get\-INet\-Cfg (IN BOOL {\em f\-Get\-Write\-Lock}, INet\-Cfg $\ast$$\ast$ {\em ppnc})}}
\label{NDI_8cpp_a2}




Definiert in Zeile 37 der Datei NDI.cpp.

Benutzt PWSTR und Release\-Obj().

Wird benutzt von Net\-Config::Install\-Net\-Service() und Net\-Config::Uninstall\-Net\-Component().



\footnotesize\begin{verbatim}39   {
40     HRESULT hr=S_OK;
41 
42     // Initialize the output parameters.
43     *ppnc = NULL;
44 
45     // initialize COM
46     hr = CoInitializeEx(NULL,
47               COINIT_DISABLE_OLE1DDE | COINIT_APARTMENTTHREADED );
48 
49     if (SUCCEEDED(hr))
50     {
51       // Create the object implementing INetCfg.
52       //
53       INetCfg* pnc;
54       hr = CoCreateInstance(CLSID_CNetCfg, NULL, CLSCTX_INPROC_SERVER,
55                 IID_INetCfg, (void**)&pnc);
56       if (SUCCEEDED(hr))
57       {
58         INetCfgLock * pncLock = NULL;
59         if (fGetWriteLock)
60         {
61           // Get the locking interface
62           hr = pnc->QueryInterface(IID_INetCfgLock,
63                       (LPVOID *)&pncLock);
64           if (SUCCEEDED(hr))
65           {
66             // Attempt to lock the INetCfg for read/write
67             static const ULONG c_cmsTimeout = 15000;
68             static const WCHAR c_szSampleNetcfgApp[] =
69               TEXT("Qbe NDI Helper");
70             PWSTR szLockedBy;
71 
72             hr = pncLock->AcquireWriteLock(c_cmsTimeout,
73                           c_szSampleNetcfgApp,
74                           &szLockedBy);
75             if (S_FALSE == hr)
76             {
77               hr = NETCFG_E_NO_WRITE_LOCK;
78 //              _tprintf(L"Could not lock INetcfg, it is already locked by '%s'", szLockedBy);
79             }
80           }
81         }
82 
83         if (SUCCEEDED(hr))
84         {
85           // Initialize the INetCfg object.
86           //
87           hr = pnc->Initialize(NULL);
88           if (SUCCEEDED(hr))
89           {
90             *ppnc = pnc;
91             pnc->AddRef();
92           }
93           else
94           {
95             // initialize failed, if obtained lock, release it
96             if (pncLock)
97             {
98               pncLock->ReleaseWriteLock();
99             }
100           }
101         }
102         ReleaseObj(pncLock);
103         ReleaseObj(pnc);
104       }
105 
106       if (FAILED(hr))
107       {
108         CoUninitialize();
109       }
110     }
111 
112     return hr;
113   }
\end{verbatim}\normalsize 
\hypertarget{NDI_8cpp_a4}{
\index{NDI.cpp@{NDI.cpp}!HrInstallNetComponent@{HrInstallNetComponent}}
\index{HrInstallNetComponent@{HrInstallNetComponent}!NDI.cpp@{NDI.cpp}}
\subsubsection[HrInstallNetComponent]{\setlength{\rightskip}{0pt plus 5cm}HRESULT Hr\-Install\-Net\-Component (IN INet\-Cfg $\ast$ {\em pnc}, IN PCWSTR {\em sz\-Component\-Id}, IN const GUID $\ast$ {\em pguid\-Class})}}
\label{NDI_8cpp_a4}




Definiert in Zeile 174 der Datei NDI.cpp.

Benutzt Release\-Obj().

Wird benutzt von Net\-Config::Install\-Net\-Service().



\footnotesize\begin{verbatim}177 {
178     HRESULT hr=S_OK;
179     OBO_TOKEN OboToken;
180     INetCfgClassSetup* pncClassSetup;
181     INetCfgComponent* pncc;
182 
183     // OBO_TOKEN specifies the entity on whose behalf this
184     // component is being installed
185 
186     // set it to OBO_USER so that szComponentId will be installed
187     // On-Behalf-Of "user"
188     ZeroMemory (&OboToken, sizeof(OboToken));
189     OboToken.Type = OBO_USER;
190 
191     hr = pnc->QueryNetCfgClass (pguidClass, IID_INetCfgClassSetup,
192                                 (void**)&pncClassSetup);
193     if (SUCCEEDED(hr))
194     {
195         hr = pncClassSetup->Install(szComponentId,
196                                     &OboToken,
197                                     NSF_POSTSYSINSTALL,
198                                     0,       // <upgrade-from-build-num>
199                                     NULL,    // answerfile name
200                                     NULL,    // answerfile section name
201                                     &pncc);
202         if (S_OK == hr)
203         {
204             // we dont want to use pncc (INetCfgComponent), release it
205             ReleaseObj(pncc);
206         }
207 
208         ReleaseObj(pncClassSetup);
209     }
210 
211     return hr;
212 }
\end{verbatim}\normalsize 
\hypertarget{NDI_8cpp_a3}{
\index{NDI.cpp@{NDI.cpp}!HrReleaseINetCfg@{HrReleaseINetCfg}}
\index{HrReleaseINetCfg@{HrReleaseINetCfg}!NDI.cpp@{NDI.cpp}}
\subsubsection[HrReleaseINetCfg]{\setlength{\rightskip}{0pt plus 5cm}HRESULT Hr\-Release\-INet\-Cfg (BOOL {\em f\-Has\-Write\-Lock}, INet\-Cfg $\ast$ {\em pnc})}}
\label{NDI_8cpp_a3}




Definiert in Zeile 130 der Datei NDI.cpp.

Benutzt Release\-Obj().

Wird benutzt von Net\-Config::Install\-Net\-Service() und Net\-Config::Uninstall\-Net\-Component().



\footnotesize\begin{verbatim}131   {
132     HRESULT hr = S_OK;
133 
134     // uninitialize INetCfg
135     hr = pnc->Uninitialize();
136 
137     // if write lock is present, unlock it
138     if (SUCCEEDED(hr) && fHasWriteLock)
139     {
140       INetCfgLock* pncLock;
141 
142       // Get the locking interface
143       hr = pnc->QueryInterface(IID_INetCfgLock,
144                   (LPVOID *)&pncLock);
145       if (SUCCEEDED(hr))
146       {
147         hr = pncLock->ReleaseWriteLock();
148         ReleaseObj(pncLock);
149       }
150     }
151 
152     ReleaseObj(pnc);
153 
154     CoUninitialize();
155 
156     return hr;
157   }
\end{verbatim}\normalsize 
\hypertarget{NDI_8cpp_a5}{
\index{NDI.cpp@{NDI.cpp}!HrUninstallNetComponent@{HrUninstallNetComponent}}
\index{HrUninstallNetComponent@{HrUninstallNetComponent}!NDI.cpp@{NDI.cpp}}
\subsubsection[HrUninstallNetComponent]{\setlength{\rightskip}{0pt plus 5cm}HRESULT Hr\-Uninstall\-Net\-Component (IN INet\-Cfg $\ast$ {\em pnc}, IN PCWSTR {\em sz\-Component\-Id})}}
\label{NDI_8cpp_a5}




Definiert in Zeile 229 der Datei NDI.cpp.

Benutzt Release\-Obj().

Wird benutzt von Net\-Config::Uninstall\-Net\-Component().



\footnotesize\begin{verbatim}230 {
231     HRESULT hr=S_OK;
232     OBO_TOKEN OboToken;
233     INetCfgComponent* pncc;
234     GUID guidClass;
235     INetCfgClass* pncClass;
236     INetCfgClassSetup* pncClassSetup;
237 
238     // OBO_TOKEN specifies the entity on whose behalf this
239     // component is being uninstalld
240 
241     // set it to OBO_USER so that szComponentId will be uninstalld
242     // On-Behalf-Of "user"
243     ZeroMemory (&OboToken, sizeof(OboToken));
244     OboToken.Type = OBO_USER;
245 
246     // see if the component is really installed
247     hr = pnc->FindComponent(szComponentId, &pncc);
248 
249     if (S_OK == hr)
250     {
251         // yes, it is installed. obtain INetCfgClassSetup and DeInstall
252 
253         hr = pncc->GetClassGuid(&guidClass);
254 
255         if (S_OK == hr)
256         {
257             hr = pnc->QueryNetCfgClass(&guidClass, IID_INetCfgClass,
258                                        (void**)&pncClass);
259             if (SUCCEEDED(hr))
260             {
261                 hr = pncClass->QueryInterface(IID_INetCfgClassSetup,
262                                               (void**)&pncClassSetup);
263                     if (SUCCEEDED(hr))
264                     {
265                         hr = pncClassSetup->DeInstall (pncc, &OboToken, NULL);
266 
267                         ReleaseObj (pncClassSetup);
268                     }
269                 ReleaseObj(pncClass);
270             }
271         }
272         ReleaseObj(pncc);
273     }
274 
275     return hr;
276 }
\end{verbatim}\normalsize 
\hypertarget{NDI_8cpp_a1}{
\index{NDI.cpp@{NDI.cpp}!ReleaseObj@{ReleaseObj}}
\index{ReleaseObj@{ReleaseObj}!NDI.cpp@{NDI.cpp}}
\subsubsection[ReleaseObj]{\setlength{\rightskip}{0pt plus 5cm}ULONG Release\-Obj (IUnknown $\ast$ {\em punk})}}
\label{NDI_8cpp_a1}




Definiert in Zeile 17 der Datei NDI.cpp.

Wird benutzt von Hr\-Get\-INet\-Cfg(), Hr\-Install\-Net\-Component(), Hr\-Release\-INet\-Cfg() und Hr\-Uninstall\-Net\-Component().



\footnotesize\begin{verbatim}18 {
19     return (punk) ? punk->Release () : 0;
20 }
\end{verbatim}\normalsize 


\subsection{Variablen-Dokumentation}
\hypertarget{NDI_8cpp_a0}{
\index{NDI.cpp@{NDI.cpp}!workState@{workState}}
\index{workState@{workState}!NDI.cpp@{NDI.cpp}}
\subsubsection[workState]{\setlength{\rightskip}{0pt plus 5cm}int \hyperlink{NDI_8cpp_a0}{work\-State}}}
\label{NDI_8cpp_a0}




Definiert in Zeile 15 der Datei NDI.cpp.

Wird benutzt von Net\-Config::Install\-Net\-Service() und Main\-Form::tmr\-Check\_\-Tick().